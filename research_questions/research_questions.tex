\documentclass{article}
\usepackage{graphicx} % Required for inserting images

\title{RCV Proposal Research Questions}
\author{mccuned }
\date{}

\begin{document}

\maketitle

\section{Research Questions}

\begin{enumerate}
\item For each election, do the methods of Borda, Condorcet, IRV, Bucklin, etc., choose the same winner? If not, why not? Does a method tend to choose a more centrist or fringe candidate?
\item What does ``fringe'' mean?
\item For each election and each method, check for the spoiler effect.
\item Are there methods which disincentivize movement toward the center in some sense? Do these finding hold true if we vary voter turnout?
\item Beef up Atkinson et al. study using CES data. Incorporate things like voters not showing up if there isn't a candidate close enough to them.
\item Check IRV for monotonicity and no-show failures.
\item For each method, how often does the method select the candidate with the third or fewest first-place votes? We study this because voters might not like such an outcome.
\item Minimax is probably the simplest Condorcet method. In three-candidate elections the method has no fairness criterion weaknesses I know of. Check how often in the 4-candidate case the method might select the Condorcet loser.
\item For each method, how safe is it for a voter to rank their favorite candidate first?
\item Analysis of top3, top4, top5 methods. Does the Condorcet winner make it through to the runoff round? Do any noticeably undeserving candidates make it to the runoff round? Are there monotonicity/strategic issues regarding the runoff round? Susceptibility to spoiler effect?
\item For everything above, use models from VoteKit to generate synthetic elections and rerun analysis.
\item Administrative challenges for various methods?

\item Look at later-no-harm criterion for Condorcet methods
\item Borda winner isn't always the majority favorite. Which elections in our data does this hold for?
\end{enumerate}

\end{document}
